\specialchapt{ABSTRACT}
Our newfound ability to analyze source code in massive software
repositories such as GitHub has led to an uptick in data-driven
solutions to software engineering problems. Source code analysis is
often realized as traversals over source code artifacts represented as
graphs. Since the number of artifacts that are analyzed is huge, in
millions, the efficiency of the source code analysis technique is very
important. The performance of source code analysis techniques heavily
depends on the order of nodes visited during the traversals: the
traversal strategy. For instance, selecting the best traversal strategy and optimizing it for a software engineering task, that infers the temporal specification between pairs of API method calls, could reduce the running time on a large codebase from 64\% to 96\%. 
While, there exists several choices for traversal strategy, like depth-first, post-order, reverse post-order, etc., there exists no technique to choose the most time-efficient strategy for traversals. In this paper, we show that a single traversal strategy does not fit all source code analysis scenarios. Somewhat more
surprisingly, we demonstrate that given the source code expressing the
analysis task (in a declarative form) one can compute static
characteristics of the task, which together with the runtime
characteristics of the input, can help predict the most time-efficient
traversal strategy for that (analysis task, input) pair. We also
demonstrate that these strategies can be realized in a manner that is
effective in accelerating ultra-large-scale source code analysis. Our
evaluation shows that our technique successfully selected the most
time-efficient traversal strategy for 99.99\%-100\% of the time and
using the selected traversal strategy and optimizing it, the running times of a
representative collection of source code analysis in our evaluation
were considerably reduced by 1\%-28\% (13 minutes to 72 minutes in absolute time) when compared against the best performing traversal strategy. The overhead imposed by
collecting additional information for our approach is less than
0.2\% of the total running time for a large dataset that contains 287K
Control Flow Graphs (CFGs) and less than 0.01\% for an ultra-large
dataset that contains 162M CFGs.