\begin{figure}[ht!]
% \centering
\begin{lstlisting}[numbers=left, tabsize=4, escapechar=@, caption={Implementation code of traversal without fixpoint.},label={lst:allNodeids}] 
public final void dfsBackward(final CFGNode node, java.util.HashMap<Integer,String> nodeVisitStatus) throws Exception {
	outputMap.put(node.id, allnodeIds(node));
	nodeVisitStatus.put(node.getId(),"visited");
	for (CFGNode pred : node.getPredecessorsList()) {
		if (nodeVisitStatus.get(pred.getId()).equals("unvisited"))
			dfsBackward(pred, nodeVisitStatus);
	}
}

public final void traverse(final boa.graphs.cfg.CFG cfg, final Traversal.TraversalDirection direction, final Traversal.TraversalKind kind) throws Exception {
	try {
		if (preTraverse(cfg)) {			
			java.util.HashMap<Integer,String> nodeVisitStatus=new java.util.HashMap<Integer,String>();
			CFGNode[] nl = cfg.nodes();
			for(int i=0;i<nl.length;i++) {
				nodeVisitStatus.put(nl[i].getId(),"unvisited");
			}
			switch(kind) {
				case TraversalKind.DFS:
				switch(direction) {
					case TraversalDirection.BACKWARD :
						dfsBackward(cfg.getExitNode(), nodeVisitStatus);
						break;
					case TraversalDirection.FORWARD :
						.....
				}
				break;					
				default:break;
			}
		}
	} catch(Exception e) {return;}
}
\end{lstlisting}
% \caption{}
% % 
% % 
% % Dominator analysis performs traversal over the graph to determine the
% % dominators of each node. It has two traversals. \textit{init} traversal collects
% % all node ids. \textit{dom\_T} is called with a fixpoint and it runs till all the
% % nodes reach the fixpoint condition.}
% \label{fig:dominator}
\end{figure}


