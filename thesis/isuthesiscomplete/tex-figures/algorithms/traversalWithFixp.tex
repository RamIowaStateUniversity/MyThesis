\begin{figure}
% \centering
\begin{lstlisting}[numbers=left, tabsize=4, escapechar=@, caption={Example traversal construct that computes post dominator.},label={lst:traversalWithfixp}] 
cfgPdom := traversal(node: CFGNode): T {
	cur_value : T;
	if(node.id==exitId) {
		self_dom:set of string;
		cur_value = self_dom;
	}
	else
		cur_value = cfgnode_ids;
	
	if(def(getvalue(node))) {
		cur_value = getvalue(node);
	}
	
	preds:=node.successors;
	foreach(i:int;def(preds[i])) {
		pred_value := getvalue(preds[i]);
		if(def(pred_value)) {
			cur_value = intersect(cur_value,pred_value);
		}
	}
	
	gen_kill := getvalue(node, allnodeIds);
	if(def(gen_kill)) {
		add(cur_value, gen_kill);
	}
		
	return cur_value;
};
\end{lstlisting}
% \caption{}
% % 
% % 
% % Dominator analysis performs traversal over the graph to determine the
% % dominators of each node. It has two traversals. \textit{init} traversal collects
% % all node ids. \textit{dom\_T} is called with a fixpoint and it runs till all the
% % nodes reach the fixpoint condition.}
% \label{fig:dominator}
\end{figure}


