\begin{figure}[ht!]
\centering
\begin{lstlisting}[numbers=left, tabsize=4, escapechar=@]
allNodes : Set<int>;

init := traversal (n : Node) { 
	add(allNodes, n.id);
}

<<<<<<< HEAD
5  dom_T := traversal (n : Node) : Set<int> { 
6 	Set<int> dom;
7	if(output(n, dom_T) == null) {
8		dom = allNodes;
9	} else {
10		dom = output(n, dom_T);
11	}
12	foreach(p : n.preds) 
13		dom = intersection(dom, output(p, dom_T)) 
14	add(dom, n.id); 
15	return dom; 
16 } 
=======
dominator := traversal (n : Node) : Set<int> { 
	Set<int> dom;
	if(output(n, dominator) == null)
		dom = N;
	else
		dom = output(n, dominator);
>>>>>>> 6f9b2a292c7913c315acccbe89d11da8c12a372c

	foreach(p : n.preds) 
		dom = intersection(dom, output(p, dominator)) 
	add(dom, n.id); 
	return dom; 
} 

<<<<<<< HEAD
22 traverse(g, init, FORWARD); 
23 traverse(g, dom_T, FORWARD, fp); 				
=======
fp := fixp(Set<int> dom, Set<int> prev_dom) : bool {
	if(equals(dom, prev_dom))
		return true;
	return false;
}

traverse(g, init, FORWARD); @\label{line:traverse}@
traverse(g, dominator, FORWARD, fp); 				
>>>>>>> 6f9b2a292c7913c315acccbe89d11da8c12a372c
\end{lstlisting}
\caption{Dominator analysis performs traversal over the graph to determine the dominators of each node. It has two traversals. \textit{init} traversal collects all node ids. \textit{dom\_T} is called with a fixpoint and it runs till all the nodes reach the fixpoint condition.}
\label{fig:dominator}
\end{figure}


