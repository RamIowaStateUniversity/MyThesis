\begin{figure}[ht!]
% \centering
\begin{lstlisting}[numbers=left, tabsize=4, escapechar=@, caption={Implementation code of traversal with fixpoint.},label={lst:cfgPdom-impNodeids}] 
public abstract class BoaAbstractTraversal<T1> {
	public java.util.HashMap<Integer,T1> outputMapObj;
	public java.util.HashMap<Integer,T1> prevOutputMapObj;
	
	public T1 getValue(final CFGNode node) throws Exception {
		return (T1)outputMapObj.get(node.getId());
	}
	
	public final void postorderBackward(final CFGNode node, java.util.HashMap<Integer,String> nodeVisitStatus) throws Exception {
		nodeVisitStatus.put(node.getId(),"visited");
		for (CFGNode succ : node.getSuccessorsList()) {
			if (nodeVisitStatus.get(succ.getId()).equals("unvisited"))
				postorderBackward(succ, nodeVisitStatus);
		}
		outputMapObj.put(node.id, cfgPdom(node));
	}
	
	public final void traverse(final boa.graphs.cfg.CFG cfg, final Traversal.TraversalDirection direction, final Traversal.TraversalKind kind, final BoaAbstractFixP fixp) {
		switch(kind.getNumber()) {
		case 12 :
			boolean fixp_flag;
			do {
				prevOutputMapObj = new java.util.HashMap<Integer,T1>(outputMapObj);
				java.util.HashMap<Integer,String> nodeVisitStatus=new java.util.HashMap<Integer,String>();
				CFGNode[] nl = cfg.nodes();
				for(int i=0;i<nl.length;i++) {
					nodeVisitStatus.put(nl[i].getId(),"unvisited");
				}
				postorderBackward(cfg.getEntryNode(), nodeVisitStatus);						
				fixp_flag=true;
				for(CFGNode node : nl) {
					boolean curFlag=outputMapObj.containsKey(node.getId());
					boolean prevFlag=prevOutputMapObj.containsKey(node.getId());
					if(curFlag) {
						if(outputMapObj.containsKey(node.getId()) && prevOutputMapObj.containsKey(node.getId())) { 
							fixp_flag=fixp_flag && fixp.invoke((T1)outputMapObj.get(node.getId()),(T1)prevOutputMapObj.get(node.getId()));
						} else {
							fixp_flag = false; break;
						}
					}
				}
			}while(!fixp_flag);
			break;
		default : break;
	}
}
\end{lstlisting}
% \caption{}
% % 
% % 
% % Dominator analysis performs traversal over the graph to determine the
% % dominators of each node. It has two traversals. \textit{init} traversal collects
% % all node ids. \textit{dom\_T} is called with a fixpoint and it runs till all the
% % nodes reach the fixpoint condition.}
% \label{fig:dominator}
\end{figure}


