\begin{figure}[ht!]
\centering
\begin{lstlisting}
2  killT := traversal (n : Node) : string { 
3  	if(n.stmt.kind == VARDECL || n.statement.kind == ASSIGNMENT) { 
4		return n.stmt.lhs; 
5	} 
6	return null; 
7  } 

8  genT := traversal (n : Node<P>) : Set<String> { 
9	return n.stmt.rhs.variables; 
10 } 


11 liveT := traversal (n : Node) : Set { 
12	live := output(n, liveT); 
13	kill := output(n, killT); 
14	gen := output(node, genT);  
15	foreach(s : node.succs) {
16		s_live := output(s, liveT);
17		live.out = union(live.out, s_live.in); 
18	}
19	live.in = live.out; 
20	remove(live.in, kill); 
21	addAll(live.in, gen); 
22	return live; 
23 }

24 fp := fixp(Set<string> out, Set<string> prev_out) : bool {
25	if(equals(out, prev_out))
26		return true;
27	return false;
28 }

29 traverse(g, killT, FORWARD, null);
30 traverse(g, genT, FORWARD, null); 
31 traverse(g, liveT, BACKWARD, fp); 				
\end{lstlisting}
\caption{Live variable analysis performs traversal over the graph to determine the live variables at each node. In \textit{kill\_traversal}, output of each node is the variable defined in it. In \textit{gen\_traversal}, output of each node is the variable used in it. The \textit{live\_analysis} traversal is the traversal which performs the live variable analysis. This traversal is called with fixpoint function as it is a data flow analysis and runs till the fixpoint is reached.}
\label{fig:live-variable}
\end{figure}
