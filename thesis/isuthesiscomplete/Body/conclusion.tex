\chapter{Conclusion}
\label{conclusion}
Improving the performance of source code analyses that runs on massive code bases
is an ongoing challenge. One way to improve the performance of
source code analysis expressed as traversals over graphs like CFGs, is by picking
the optimal traversal strategy that defines the order of nodes visited. The
selection of the best traversal strategy depends both on the properties of the
analysis and the input graph on which the analysis is run.
We proposed a hybrid technique for selecting and optimizing graph traversal
strategies for source code analysis expressed as traversals over graphs. Our
solution includes a system for expressing source code analysis as traversals, a set
of static properties of the analysis and algorithms to compute them, a decision
tree that checks static properties along with graph properties to select the
most time-efficient traversal strategy.
Our evaluation shows that the hybrid technique successfully selected the most
time-efficient traversal strategy for 99.99\%--100\% of the time and 
using the selected traversal strategy and optimizing it, the running times of a
representative collection of source code analysis in our evaluation
were considerably reduced by 1\%-28\% (13 minutes to 72 minutes in absolute time) when compared against the best performing traversal strategy. The case studies show that hybrid traversal reduces 80--175 minutes in running times for three software engineering tasks. The overhead imposed by 
collecting additional information for our approach is less than 0.2\% of 
the total running time for a large dataset and less than 0.01\% for an 
ultra-large dataset.