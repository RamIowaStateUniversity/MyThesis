\chapter{Related Work}
\label{sec:related}

To the best of our knowledge, our proposal to leverage information about the
program analysis code, and the nature of the data on which analysis is applied
to select appropriate traversal strategies has not been explored previously.
Below we discuss works that are related to various aspects of our proposal. 

\section{Mixing static and dynamic information.} %
The general philosophy of mixing static and dynamic information has a long 
history(~\cite{Ernst2003}) in both the software engineering and the programming 
languages communities, with examples such as DSD-Crasher(~\cite{DSD-Crasher}), 
Palus(~\cite{Palus}), segmented symbolic execution(~\cite{Le2013}), guided 
dynamic symbolic execution(~\cite{Christakis2016}), gradual typing(~\cite{Siek07})
, hybrid type checking (\cite{Flanagan06}), intensional polymorphism(~\cite{Harper95), 
Crary98}, etc. While our proposal also mixes static information 
about program analysis with dynamic information about the data, none of the 
previous works have proposed utilizing this information for selecting 
appropriate traversal strategies for realizing the program analysis.

\section{Optimizing program analysis.}%
Atkinson ~\cite{atkinson2001implementation} presented techniques that reduce 
the time and space required to perform data-flow analysis of large programs. 
While their techniques proposed modifications to the underlying data-flow 
analyses that yield improvement in performance and also proposed reclamation 
of the data-flow sets during data-flow analysis that result in saving space, 
hybrid approach gives performance gain by analyzing the user written 
algorithm and the input graph received.\newline Kildall ~\cite{kildall1973unified} 
presented an algorithm which, in conjunction with various optimizing 
functions, provides global program optimization, Optimizing functions have 
been described which provide constant propagation, common sub-expression 
elimination, and a degree of register optimization. While their approach 
provides unified approach to global program optimization, we concentrate on 
optimizing the process that does program optimization using the program's 
structure and the algorithms characteristics.

\section{Ultra-large-scale source code mining.}%
In terms of ultra large scale processing, Boa(~\cite{dyer2013boa}) is a 
language and infrastructure for analyzing ultra-large-scale software 
repositories. Boa provides a different kind of performance gain through its 
infrastructure and eases testing MSR-related hypotheses, it is not suitable 
for graph processing algorithms and does not leverage information from 
algorithms written in Boa.
~\cite{upadhyaya2017accelerating} also tried to accelerate Ultra large scale score code mining. Their key idea is to analyze the interaction pattern between the mining task and the artifact to cluster artifacts such that running the mining task on one candidate artifact from each cluster is sufficient to produce results for other artifacts in the same cluster. Their artifact clustering criteria go beyond syntactic, semantic, and functional similarities to mining-task-specific similarity, where the interaction pattern between the mining task and the artifact is used for clustering. While their approach does task-specific clustering and extrapolates results, we try to analyze the analysis and come up with the best way to traverse the graph so that we can finish the analysis by visiting lesser number of nodes and lesser operations.
~\cite{dyer2013declarative} developed domain-specific language abstractions for easily writing source code mining tasks on billions of AST nodes. While their language abstractions were for AST nodes, our language features were for traversing CFGs. Another important difference is that we can specify a user defined fixpoint for the traversals and the traversal will run till the fixpoint is reached. We can also provide the direction of traversal and the traversal can return outputs while the visitor construct in ~\cite{dyer2013declarative} does not.
\section{Graph traversal optimization.}%
There have been many works that targeted graph traversal optimization through 
various ways. Green-Marl(~\cite{hong2012green}) provides performance benefits 
by using domain specific knowledge in applying optimizations. It uses high-level 
algorithmic description written in Green-Marl to exploiting the exposed 
data level parallelism. While green marl provides performance benefits by 
taking algorithm written into consideration, hybrid approach takes both 
algorithm and graph structure into account. And in ultra large scale dataset, 
containing millions of graphs with different structures, the gain that we can 
incur by taking graph structure into account is significant.

Pregel(~\cite{malewicz2010pregel}) is a MapReduce like framework that aims to 
bring distributed processing to graph algorithms. While Pregel's performance 
gain is through parallelism and handles large graphs processing through 
vertex centric approach, our approach achieves performance gain by traversing 
the graph efficient suitable to the algorithm.

There have also been few libraries that support parallel or distributed graph 
analysis: Parallel BGL(~\cite{gregor2005parallel})  is a distributed version of 
BGL while SNAP(~\cite{bader2008snap})  is a stand-alone parallel graph analysis 
package.
% TODO - text below requires huge amount of editing.
% Organize under the categories given above.
