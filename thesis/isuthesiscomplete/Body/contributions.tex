\chapter{Contributions}

In this work, we develop {\em hybrid traversal selection}, a novel program
analysis optimization technique for BigCode analyses expressed as graph traversals.
Our approach relies on our observations that a suitable traversal strategy is
dependent on both the program analyses and input graphs' properties.
The former are static properties and the latter are dynamic properties.
More importantly, depending on the properties of analyses and graphs, existing
traversals could be optimized to improve their performance.  
Hybrid traversal selection relies on several technical underpinnings: 

\noindent{\bf [Traversal Declaration and Traverse Expression]} 
Programmers can declare their program analyses
as one or more {\bf traversal} declarations and run them using {\bf traverse} expression.
The runtime implementation of the {\bf traverse} expression selects a suitable 
traversal strategy based on the {\bf traversal} declaration and the input graph.
Main benefit of these linguistic abstractions is that they abstract away 
traversal related code so that the traversal strategy can be replaced as needed 
by the analysis runtime.

\noindent{\bf [Data-Flow and Loop Sensitivity Analyses for Traversals]}
We show that traversal strategy selection depends on three critical properties
of the traversal: {\em data-flow sensitivity}, {\em loop sensitivity}, and 
{\em traversal direction}. We propose algorithms for computing these properties.
Our analysis system implements these algorithms.
These properties are computed statically and their values are stored as metadata 
to be utilized by the traversal selection at runtime.

\noindent{\bf [Graph \Graphprop{}]}
We have observed that the traversal strategy selection depends on one
dynamic property of the input which is {\em graph \graphprop{}}. This property partitions
the set of graphs into three categories: those that are sequential, those with branches
but no cycles, and those with cycles. Our system computes this property at 
graph construction time and stores it as an attribute in the runtime graph representation. 

\noindent{\bf [Decision Tree for Traversal Strategy Selection]}
We have devised a {\em decision tree} for traversal strategy selection that 
given data-flow sensitivity, loop sensitivity, and traversal direction properties 
of the analysis and the \graphprop{} property of the input graph produces 
a selection for traversal strategy.
While the tree is utilized by our automated system, it could also be used by 
a programmer for manual traversal selection.

Hybrid traversal selection has two direct benefits. First, it improves the 
efficiency of BigCode analysis thus speeding up data-driven science in this important area.
Second, it frees up programmers from having to write traversal related code 
and then optimizing it based on the analysis and the graph at hand. 

We have evaluated our technique using a set of 21 source code analysis that
includes control and data-flow analysis, and analysis to find bugs. The
evaluation is performed on two datasets: a dataset containing well-maintained
projects from DaCapo benchmark (contains a total of 287K graphs), and a
ultra-large dataset containing more than 380K projects from GitHub (contains a
total of 162M graphs). Our evaluation shows that our technique successfully selected
the most time-efficient traversal strategy for 99.99\%--100\% of the time and
using the selected traversal strategy and optimizing it, the running times of a
representative collection of source code analysis in our evaluation
were considerably reduced by 1\%-28\% (13 minutes to 72 minutes in absolute time) when compared against the best performing traversal strategy. The case studies show that hybrid traversal reduces 80--175 minutes in running times for three software engineering tasks.
The overhead imposed by our approach is negligible (less than 0.2\% of the total 
running time for a large dataset and less than 0.01\% for an ultra-large dataset). 

In summary, this paper makes the following contributions:
\begin{itemize}
	\item It describes a system for expressing source code analysis as traversals. The
	constructs and operations in the system allows different source code analysis to
	be expressed in a manner that allows automatic selection of the best traversal
	strategies.
	\item It defines a set of novel properties about the traversal expressed 
	in our system. It also describes algorithms for analysing traversals for inferring 
	these properties. 
	\item It describes a novel decision tree for selecting the most suitable traversal 
	strategy. The static and runtime properties also allows
	certain optimizations to be performed on the selected traversal strategy to
	further improve the performance.
	\item It demonstrates the potentially broad range of applications of
	hybrid traversal selection for optimizing source code analysis such
	as available expressions, local may alias, live variable, nullness
	analysis, post dominator, reaching definitions, resource status,
	very busy expression, etc.
\end{itemize} 

% The rest of the paper is organized as follows: the next section
% (\secnref{sec:language}) describes our system for expressing program analysis as
% traversals, followed by an overview of our approach in \secnref{sec:overview}.
% Key details about our hybrid technique is provided in \secnref{sec:approach}.
% Section \secnref{sec:eval} presents our evaluation, followed by a discussion of
% the key related works in \secnref{sec:related} and a conclusion.
% 
% 
% The next section describes the constructs that we are going to use describe
% our analysis. This construct will in-turn be used in our algorithms to
% analyze our analysis.

% 
% Graph traversal is a well-studied problem, and various traversal strategies
% exist e.g. the postorder, reverse postorder, worklist-based traversal, etc.
% These strategies significantly outperform the alternative when applied to the
% kind of program artifacts that suits them well.
% On the other hand, adopting a traversal that is not suitable to the artifact may
% lead to more traversals over the graphs and more operations per node for the
% analysis. \newline
% 
% Software repositories like  SourceForge (350,000+ projects), GitHub (250,000+
% projects), and Google Code (250,000+ projects) contain massive codebases, also
% called Big code.
% Big code mining as its most basic operation contains huge number of traversals
% of the AST and other derived artifacts such as the control flow graph (CFG), the
% data-flow graph (DFG), etc. These artifacts are of different kinds. So,
% selecting a single traversal strategy may not be desirable. And the mining task
% that needs to be performed on such large scale data can have various
% characteristics that they may warrant a particular strategy. Selecting a
% in-appropriate traversal may lead to significant increase in execution time and
% for such large scale software repository
% analysis~\cite{bajracharya2014sourcerer},
% ~\cite{bevan2005facilitating},~\cite{gousios2009alitheia},~\cite{gousios2012ghtorrent}
% this could be costly.\newline
% 
% Our hybrid approach is based on the intuition that the analysis task that needs
% to be performed can have different characteristics and the large scale data that
% the analysis has to be performed upon contains artifacts of different
% characteristics. These information may lead us to the traversal strategy best
% suited for the analysis and the artifact. Leveraging this information from
% analysis code statically and information from artifact dynamically, we can apply
% the best traversal that needs fewer node visits and fewer operations per node to
% perform the analysis when compared to other approaches.\newline
% 
% We have applied our approach to twenty traversal-based formulation of fairly
% standard program analyses, showing broad applicability of our ideas. For these
% analyses we see an improvement between 10-40\% over existing traversal
% strategies in terms of the total execution time. Most importantly, hybrid
% approach improves performance significantly from worklist approach which is the
% most used approach for solving data flow analysis. We achieve these performance
% gain without compromising the soundness of the result. Last but not least, we
% show very high accuracy in traversal selection between 99-100\% owing to our
% selection criteria and algorithms. We believe that our approach can help
% decrease the turnaround time of data-driven experiments in this area.
% 
% We summarize the other contribution of this paper to realize hybrid traversal
% selection:
% \begin{itemize}
% 	\item Proposed efficient way of identifying the static properties(data flow sensitivity, loop sensitivity) from the analysis code.
% 	\item A decision tree to identify the best traversal strategies among the candidates based on the static properties of the traversal and the dynamic property of the graph.
% 	\item Identified the optimization that could be performed on the selected traversal strategy that makes it the traversal that has the best execution time among the candidates for the given analysis and the graph.
% 	\item Introduced a traversal property called "Loop sensitivity" for data flow analysis that identifies whether the loops present in the graph introduces more traversal of the graph to reach fixpoint.
% 	%\item Identified the characteristics of the analysis and the input graph for which fix point checking is redundant for data flow analysis.
% 	%\item Proposed efficient way of identifying the dynamic properties(type of control flow) from the CFG.
% \end{itemize}
